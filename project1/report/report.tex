\documentclass[conference]{IEEEtran}

\usepackage{listings}

\usepackage{c/style} % include custom style for C.
\lstset{basicstyle=\footnotesize\ttfamily,
  breaklines=true}




\begin{document}

\title{Performance enhancement using CUDA in a simulation of heat diffusion}

\author{
  \IEEEauthorblockN{Gonçalo Lourenço \\ nº55780 \\ gm.lourenco@campus.fct.unl.pt}
  \and
  \IEEEauthorblockN{Joana Faria \\ nº55754 \\ js.faria@campus.fct.unl.pt}
}

\maketitle



\section{Introduction}
This assignment aims to optimize a base code, written in \texttt{C}, that computes a simulation for heat diffusion. For the optimization, we seek to take advantage of GPU programming using \texttt{CUDA}, and explore several alternatives to find the most efficient one.

To find the best performance we will explore different approaches, namely analyzing different configurations, exploring the impact of using shared memory, and the difference between using streams and not.

Knowing that the architecture of the system influences the performance obtained, the first step of our work is to understand the architecture of the system where our program will run. So we present the characteristics below:
\lstinputlisting{infoDevice}

\section{Results}
To accurately compare the different versions all the tests were performed in the university cluster and averaged the execution times over ten execution. The parameters chosen were:
\lstinputlisting[language=C,style=c, firstline=71, lastline=76]{../proj1/main.c}


\subsection{Sequential Verion}
We start by analyzing and profiling the sequential version to understand what improvements can be done, and what parts are the more problematic.

By a quick analysis of the code, we expect the cycle present in the \texttt{main} function to be the biggest problem. This assumption is supported by the reports of the profiling tools. The output of these tools can be seen in the files \texttt{results/sequential/gprof} and \texttt{results/sequential/perf}.










\end{document}